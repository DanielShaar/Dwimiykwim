\documentclass[11pt]{article}
\usepackage[charter]{mathdesign}
\usepackage{amsmath, verbatimbox}

\newtheorem{lemma}{Lemma}

\title{6.945 Final Project \\ \emph{Dwimiykwim}}
\author{Ziv Scully (\texttt{ziv@mit.edu}) \\
  Daniel Shaar (\texttt{dshaar@mit.edu})}
\date{}

\begin{document}
\maketitle


\section{Tagging}

The first piece of this project that is intended to set groundwork
for inference and matching, is a lightweight system for tagging data.
The motivation for tagging data is that it creates a method for
describing data with useful characteristics that can be used for
performing argument inference with tag checks as predicates.

Tagged data consists of a value and a list of tags,
which can be used to describe the significance of the data.
We can add more tags to data as it moves through a program,
and remove the tags if we would just like the value.
However, we can always perform operations on tagged data
without removing the tags in advanced because procedures by default
untag data before processing it.

Since MIT Scheme does not come with a nice type system by default,
tags allow us to work with variables by more generally describing their
characteristics. As stated earlier, the primary usefulness of these tags
lies in argument inference.
By giving values tags, we can match tags to determine if a variable
is an appropriate argument to a function.
Functions can request that certain variables have certain tags,
or more generally, that these variables satisfy predicates.
This allows us to verify that the user is performing his intentions
by forcing him/her to think in advance about what really belongs
in the function call.

Although this tag system does not have to be used in order to perform
the core features of Dwimiykwim, argument inference,
and out of order operations, it is convenient to utilize it
in many cases where the predicate one checks describes a characteristic
of the data.


\section{Argument Inference}

We have written an interpreter for a Scheme-like language
that supports automatic inference of procedure arguments.
The core of the argument inference system is in three new special forms:
\texttt{madlab}, \texttt{madblock}, and \texttt{infer}.

The \texttt{madlab} special form is a variation of \texttt{lambda}.
Specifically, the form creates a procedure
that can take its arguments in any order.
Instead of using position to match given arguments
to the variables they are bound to,
a \texttt{madlab} specifies a predicate for each input variable
that the argument bound to that variable must satisfy.
For example,
\begin{verbbox}
(define madmap
  (madlab ((xs list?) (f procedure?))
    (map f xs)))
\end{verbbox}
\begin{center}\theverbbox\end{center}
is a variant of \texttt{map} that can take its arguments in either order.
We call the resulting procedures ``madlab procedures'' or simply ``madlabs''.
For most purposes, madlabs are ordinary compound procedures
that happen to have unusually flexible interfaces.
For example, if we define
\begin{verbbox}
(define (curry f . args)
  (lambda more-args
    (apply f (append args more-args))))
\end{verbbox}
\begin{center}\theverbbox\end{center}
then both \texttt{(curry madmap exp)} and
\texttt{(curry madmap (iota 16))} work as expected,
raising $e$ to each of the elements of an input list
and mapping an input function over the integers from 0 to 15, respectively.
Note that the argument matching is not done until the madlab is applied.
To demonstrate this, consider
\begin{verbbox}
(define silly
  (madlab ((xy (member-of '(x y))) (yz (member-of '(y z))))
    (list xy yz)))
\end{verbbox}
\begin{center}\theverbbox\end{center}
and the partial application \texttt{(curry silly 'y)},
where \texttt{member-of} has the obvious namesake meaning.
When it's applied to \texttt{'x},
the \texttt{'y} is matched with \texttt{yz},
but when it's applied to \texttt{'z},
the \texttt{'y} is matched with \texttt{xy}.

The \texttt{madlab} special form is a variation of \texttt{begin}.
Just like \texttt{begin},
\texttt{madblock} groups together a sequence of expressions,
evaluates each of them in turn,
and returns the result of the last one.
The only difference is that the result of each evaluation in the sequence
is added to an \emph{inference context},
which, as we will soon explain,
is the list of values availabe to \texttt{infer}.
The inference context is dynamically bound
(the interpreter uses MIT Scheme's \texttt{fluid-let})
to the empty list at the beginning of each \texttt{madblock},
so the inference context is empty at the start of the sequence.
Our interpreter makes it easy to refer explicitly
to values of expressions earlier in the sequence
using the \texttt{define} special form
by having \texttt{define} return the result of evaluating its body
as opposed to an unspecified value or the name it was bound to.

The \texttt{infer} special form is what Dwimiykwim is all about.
It takes a madlab and any number of other expressions
and applies the madlab to the given expressions
plus any additional necessary values from the inference context.
That is, as its name suggests,
\texttt{infer} infers what arguments to pass to the given madlab.
For instance,
\begin{verbbox}
(madblock
 (curry list 'say)
 'dont-pick-me-im-a-symbol-not-a-procedure-or-list
 "a very distracting string"
 '(1 2 5 3-sir 3)
 (infer madmap))
\end{verbbox}
\begin{center}\theverbbox\end{center}
returns \texttt{((say 1) (say 2) (say 5) (say 3-sir) (say 3))}.
Inference only succeeds if there is an unambiguous matching
between the madlab's predicates and the union of the given expressions
and values from the inference context
with the additional constraint that every given expression is matched.
(See Section \ref{bipartite} for details.)
For example, only the first two inferences in
\begin{verbbox}
(madblock
 (curry list 'say)
 (define xs '(1 2 5 3-sir 3))
 (infer madmap)
 (infer madmap xs)
 (infer madmap))
\end{verbbox}
\begin{center}\theverbbox\end{center}
will succeed.
A single procedure, \texttt{(curry list 'say)},
is in the inference context the whole time.
When the first \texttt{infer} happens,
there is also only one list, so inference succeeds.
Note, however, that the resulting list is added to the inference context.
The second \texttt{infer} is explicitly passed a list,
and inference succeeds thanks to this constraint.
By the time the third \texttt{infer} happens
there are three lists in the inference context,
so inference fails due to ambiguity.

Just as the body of a \texttt{lambda} with multiple expressions
desugars to a single \texttt{begin} expression,
the body of a \texttt{madlab} desugars to a single \texttt{madblock}.
Additionally, each of the argument variables is added as an expression
to the beginning of the madblock.
The effect of this is that \texttt{infer} can be used freely
in the body of a \texttt{madlab}
and the arguments are automatically added to the inference context.
If for some reason we need to keep the arguments out of the context,
we can write body as a \texttt{madblock} explicitly.
There is nothing to worry about with regards to nesting
because each \texttt{madblock} starts with a fresh inference context.
In fact, the intended style is for \texttt{madblock} to be rare
and mostly invoked implicitly through \texttt{madlab}.
This is because \texttt{madlab} creates a new environment,
which is a good idea given the intended synergy with \texttt{define}.


\section{Bipartite Matching}\label{bipartite}

Inferring arguments of madlabs
reduces to the following problem in graph theory.
We are given a bipartite graph with vertex partitions $A$ and $B$
satisfying $|A| \leq |B|$
along with a subset $B^* \subseteq B$ of ``required'' vertices,
and we ask whether there exists a unique matching of size $|A|$
such that every vertex of $B^*$ is matched.
$A$ is the set of predicates of the madlab's arguments,
$B^*$ is the set of values passed in explicitly,
$B$ is the union of $B^*$ and the set of values in the inference context,
and there is an edge between $a \in A$ and $b \in B$
if and only if $b$ satisfies $a$.

This problem is quite easily solved by a variation on
the traditional maximum bipartite matching algorithm,
which we review quickly here.
We refer to vertices in $A$ as ``sources'' and vertices in $B$ as ``targets''.
Let $E$ be the edge set of the graph and $M \subseteq E$ be a matching.
We think of edges in $M$ as being oriented from $B$ to $A$
and edges in $E \setminus M$ as being oriented from $A$ to $B$.
An \emph{augmenting path} of $M$ is
a path from an unmatched source to an unmatched target
that follows only edges in $E \setminus M$ from sources to targets
and only edges in $M$ from targets to sources.
Given an augmenting path of a matching,
swapping the orientations
(that is, inserting or removing from the matching as appropriate)
of all the edges in the path
yields a larger matching:
all intermediate vertices in the path remain matched,
but the previously unmatched endpoints are now matched.
This means a maximum matching has no augmenting paths.
It is well-known that the converse also holds:
if a matching has no augmenting path, it is not maximal.
Therefore, to find a maximum matching,
we repeatedly search for augmenting paths,
flipping edge orientations when we find them,
until there are no more,
at which point the edges from $B$ to $A$ are the edges of the maximum matching.

Our problem differs from traditional maximum matching in two ways.
First, we require that some targets $B^* \subseteq B$ be matched
because we must use every argument that was passed in explicitly.
To do this, we start by finding a maximum matching $M$ of $B^*$ with $A$,
reporting failure if $|M| < |B^*|$,
and then matching $A$ with $B$ using $M$ as a starting point,
guaranteeing that all of $B^*$ will remain matched.
Second, we are concerned with whether the maximum matching is unique
because there must not be multiple ways
to match all the predicates with arguments.
To do this, given a maximum matching $M$,
for each edge $e \in M$,
we attempt to find an augmenting path of the matching $M \setminus \{e\}$
in a graph with reduced edge set $E \setminus \{e\}$,
with the additional requirement that
if $e$ was incident with a vertex $b \in B^*$
then the augmenting path must finish at $b$.
That these algorithms are valid follows from the fact that,
given a non-maximum matching $M$,
there is a maximum matching containing a vertex unmatched by $M$
only if there is an augmenting path of $M$ containing that vertex,
which is a slight strengthening of the result mentioned earlier.

Once the graph is constructed, all of this can be done in quadratic time.
Our implementation is purely functional
and makes liberal use of linear-time list procedures,
so it is slower than this by approximately another quadratic factor,
but procedures generally don't take in more than, say, 8 arguments,
and $O(8^4)$ is perfectly acceptable running time for our prototype.


\section{Debugging}\label{debugging}

In addition to constructing mechanisms for tagging,
inferring arguments in function calls, and matching function arguments,
we have created a way for the user to easily correct ambiguity errors
that arise from more than one possible way to interpret an inference.
For example, consider the simple \textit{madlab} below:
\begin{verbbox}
(define num-str
 (madlab ((x number?) (y string?))
         (list y x)))
\end{verbbox}
\begin{center}\theverbbox\end{center}
If we were to call a \textit{madblock} that had two number values and strings,
then upon inference, we would have an ambiguity error.
To handle this case, we put the program into debug mode,
and display the context, edges, and required args to the user.
The context is indexed, so that the user does not have to spend time
typing in every expression they would like to force in the matching.
Using this information, the user is then expected to add new required args
to settle any ambiguities.
The reason we ask for required arguments is because those will be used
in the matching and will usually eliminate multiple matchings.
Once the user enters some combination of unambiguous args,
we notify the user of the successful matching and exit debug mode.
A sample workflow for \textit{num-str} would be:
\begin{verbbox}
(madblock
 1
 "foo"
 2
 "bar"
 (infer num-str))

=== Dwimiykwim Tawimiydkwim ===

Context:
(0 "bar")
(1 2)
(2 "foo")
(3 1)

Edges:
(x 1 3)
(y 0 2)

Required:
()

New required:
(0)
Ambiguous matching!

New required:
(0 3)
Unambiguous matching! Terminate debugging mode!
\end{verbbox}
\begin{center}\theverbbox\end{center}
In this example, the user was allowed to test multiple sets of required
arguments, until one achieved a good matching.
The user would then be expected to go and correct the code
by specifying those as required arguments in the inference.
The new program would now be:
\begin{verbbox}
(madblock
 1
 "foo"
 2
 "bar
 (infer num-str 1 "bar"))
\end{verbbox}
\begin{center}\theverbbox\end{center}
Since the user is providing so many arguments, it would not be very useful
to use inference unless they modified the code to remove some expressions.

Under the hood of the debugger, when we encounter a bad matching,
we enter debug mode, indexing each item in the context,
and print all the information we know about the matching that had an error.
We ask the user to give us a list of newly required arguments,
verifying that this is indeed a list of numbers,
and checking using the matching function if the ambiguity resolves.
In the case where the user has not specified enough required arguments,
we ask the user for a new set of required arguments,
not saving the initial choice.
This continues until either the user exits the debugger,
or a good match is found and program terminates with an error.
\end{document}
